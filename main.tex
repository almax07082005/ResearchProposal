\documentclass[oneside,final,12pt,a4paper]{extreport}
\renewcommand{\thetable}{\Roman{table}}\usepackage{hyperref}
\usepackage{hyperref}

\hypersetup{
    colorlinks=true,
    linkcolor=blue,
    filecolor=magenta,      
    urlcolor=cyan,
    citecolor=cyan,
    pdftitle={Overleaf Example},
    pdfpagemode=FullScreen,
}
\usepackage[utf8]{inputenc}
\usepackage{graphicx}
\usepackage[affil-it]{authblk}
\usepackage[english]{babel}
\usepackage[backend=biber,style=ieee,autocite=inline]{biblatex}
\bibliography{ref.bib}
\DefineBibliographyStrings{english}{%
  bibliography = {References},}
\usepackage{blindtext}


\usepackage{amsmath,amsfonts}
\usepackage{url}
\usepackage{titlesec}
\usepackage{csquotes}
\usepackage{longtable}
\titleformat{\section}[hang]{\fontsize{20}{24}\selectfont\filcenter}{\Roman{section}}{1em}{}
\titleformat{\subsection}[hang]{\itshape}{\Alph{subsection}.}{1em}{}[]
\titleformat{\subsubsection}[runin]{\itshape}{\arabic{subsubsection})}{1em}{}[$:$]
\titlespacing{\subsubsection}{1em}{1em}{1em}
\titleformat{\paragraph}[runin]{\itshape}{\alph{paragraph})}{1em}{}[$:$\quad]
\titlespacing{\paragraph}{2em}{1em}{1em}
\usepackage{vmargin}
\setpapersize{A4}
\setmarginsrb{2.5cm}{2cm}{2cm}{2cm}{0pt}{10mm}{0pt}{13mm}
\usepackage{setspace}
\sloppy
\setstretch{1.5}
\usepackage{indentfirst}
\parindent=1.25cm
\usepackage{caption}
\usepackage{subcaption}
\DeclareCaptionLabelSeparator{figSep}{.\quad}
\captionsetup[figure]{labelfont={normalfont}, name={Fig.}, labelsep=period}
\captionsetup[table]{labelfont={normalfont}, name={TABLE}, labelsep={newline}}
\title{Research Proposal}
\author{Ilona, Egor, Maksim}
 
\affil{Innopolis University}

\begin{document}

\maketitle

\newpage

\section{Introduction/Background and Rationale (choose any)}
Please, state the problem and explain its significance. Clearly indicate your research goal, research question(s), or hypothesis/-es. Refer to the existing literature as need be. 

Your Introduction can also add definitions of terms if you think you use any that can be confusing to the reader (optional). You can also add the limits you assign to your research (optional) here. You can use subsections to make your writing clearer. Any number of subsections can be used or no subsections at all.
\subsection{Title of subsection 1}

This is how you cite a figure in your text: \textit{Fig. \ref{fig:inno} shows a logo of Innopolis University.}

\begin{figure}[htp]
    \centering
    \includegraphics[width=6cm]{figs/inno.png}
    \caption{Image of IU logo.}
    \label{fig:inno}
\end{figure}Example of referencing equations: 



\begin{equation}\label{my_first_eqn}
e=mc^2
\end{equation}

\begin{equation}\label{my_second_eqn}
e=fc^2
\end{equation}

This is now you cite equations in your text: \textit{Equation (\ref{my_first_eqn}) shows a well-known relation}.
\subsection{Title of subsection 2}
Use subsections to make your writing clearer.
\section{Literature Review}

In this section, refer to the existing literature on your topic. Discuss main theories, concepts, controversies, and approaches. Do not simply list the sources. Sources synthesis is required: compare and contrast the literature. Once you have done that, interpret the information in the sources. In other words, tell your readers how to understand the facts provided by the cited sources. Make sure that you have revealed a research gap.

Any number of subsections can be used or no subsections at all.

See Examples of references below.
\subsection{Referring to conference proceedings}
\label{sec:subsection1}

This is how you refer to online conference proceedings \cite{FosterEtAl:2003}. Note that you have to abbreviate the conference titles as discussed in Pages 6-7 of \href{https://journals.ieeeauthorcenter.ieee.org/wp-content/uploads/sites/7/IEEE_Reference_Guide.pdf}{the IEEE Reference Guide}.

\subsection{Referring to journals}
\label{sec:subsection2}

Here is an example of an online journal with one author and no DOI \cite{Poock2002GraduateSO}. Note that you have to abbreviate the journal titles as discussed in Section IV of \href{https://journals.ieeeauthorcenter.ieee.org/wp-content/uploads/sites/7/IEEE_Reference_Guide.pdf}{the IEEE Reference Guide}.

Here is an an example of an online journal with two authors and DOI \cite{Hullinger2014STUDENTAE}.

Here is an an example of an online journal with two authors and no DOI \cite{Oswalt2007WhatTD}.

Here is an an example of an online journal with three authors and DOI \cite{PARK2022144}.

Here is an example of an online journal with an article number instead of page numbers \cite{HERNANDEZRUIZ2022101953}.

Here is an example of an online journal with italics and capitalization in the title \cite{BAKHRIANSYAH2022101114}.

Here is an example of how you refer to online journals with six authors or more \cite{VACHON202121}, \cite{Stewart}.

Here is an example of how you refer to an online journal written in Russian \cite{Bulatov}.

\subsection{Referring to books}
\label{sec:subsection3}

This is how you refer to a book \cite{Mazza}.

This is how you refer to a book chapter when the entire book was written by the same author(s) and the chapter of interest is identified by a chapter number \cite{peyret2012:ch7}.

This is how you refer to a book chapter when the chapter has its own author and title \cite{Mihalcea:2006}.

This is how you refer to a book available online \cite{Hatch}.

\subsection{Referring to miscellaneous sources}
\label{sec:subsection4}
This is how you refer to a web site \cite{web:lang:stats}.

This is how you refer to a direct quotation: According to \cite[ 110]{FosterEtAl:2003}, "a subject of current interest in the machine translation (MT) world ... is the rapid development of systems for novel language pairs."

\subsection{Title of subsection 1}
Use subsections to make your writing clearer. Example of a table.

\begin{longtable}{c|c|c}
\caption[This is the title I want to appear in the List of Tables]{\textsc{This Is a Table Example}} \label{tab:pfams} \\
\hline
A & B & C \\
\hline
\endfirsthead
\multicolumn{3}{@{}l}{} \\
\hline
A & B & C\\
\hline
\endhead
a1 & b1 & c1 \\
a2 & b2 & c2\\
a3 & b3 & c3\\
a4 & b4 & c4\\
\hline
\end{longtable}
This is how you cite a table in your text: \textit{Table \ref{tab:pfams} contains some random data.}


\subsection{Title of subsection 2}
Use subsections to make your writing clearer.

\section{Methods/Methodology/Procedure/Research Design (choose any)}
Provide reasoning behind the proposed methods/methodology/procedure. Refer to the literature about the use of methods in similar studies. 

Discuss a research approach that you are adopting (inductive or deductive). Assess time, resources, and skills required to complete your research. Describe how you are going to collect data and prepare them. List the limitations of the suggested methodology and provide ways to overcome them.


Any number of subsections can be used or no subsections at all.
\subsection{Name of subsection 1}
Use subsections to make your writing clearer.
\subsection{Title of subsection 2}
Use subsections to make your writing clearer. 
\section{Anticipated Results}
Discuss what results you anticipate. 

Any number of subsections can be used or no subsections at all.
\subsection{Title of subsection 1}
Use subsections to make your writing clearer.
\subsection{Title of subsection 2}
Use subsections to make your writing clearer.
\section{Discussion}
Discuss the possible contribution of your research to the field. Connect the expected results to the research results in similar studies. Refer to the literature about the similar studies. Mention the possible limitations of the study and ways to overcome them.

Any number of subsections can be used or no subsections at all.
\subsection{Title of subsection 1}
Use subsections to make your writing clearer.
\subsection{Name of subsection 2}
Use subsections to make your writing clearer.
\newpage
\printbibliography[heading=bibintoc,title={References}]

\end{document}
