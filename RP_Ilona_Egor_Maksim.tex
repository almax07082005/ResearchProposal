\documentclass[oneside,final,12pt,a4paper]{extreport}
\renewcommand{\thetable}{\Roman{table}}\usepackage{hyperref}
\usepackage{hyperref}

\hypersetup{
    colorlinks=true,
    linkcolor=blue,
    filecolor=magenta,      
    urlcolor=cyan,
    citecolor=cyan,
    pdftitle={Overleaf Example},
    pdfpagemode=FullScreen,
}
\usepackage[utf8]{inputenc}
\usepackage{graphicx}
\usepackage[affil-it]{authblk}
\usepackage[english]{babel}
\usepackage[backend=biber,style=ieee,autocite=inline]{biblatex}
\bibliography{ref.bib}
\DefineBibliographyStrings{english}{%
  bibliography = {References},}
\usepackage{blindtext}


\usepackage{amsmath,amsfonts}
\usepackage{url}
\usepackage{titlesec}
\usepackage{csquotes}
\usepackage{longtable}
\titleformat{\section}[hang]{\fontsize{20}{24}\selectfont\filcenter}{\Roman{section}}{1em}{}
\titleformat{\subsection}[hang]{\itshape}{\Alph{subsection}.}{1em}{}[]
\titleformat{\subsubsection}[runin]{\itshape}{\arabic{subsubsection})}{1em}{}[$:$]
\titlespacing{\subsubsection}{1em}{1em}{1em}
\titleformat{\paragraph}[runin]{\itshape}{\alph{paragraph})}{1em}{}[$:$\quad]
\titlespacing{\paragraph}{2em}{1em}{1em}
\usepackage{vmargin}
\setpapersize{A4}
\setmarginsrb{2.5cm}{2cm}{2cm}{2cm}{0pt}{10mm}{0pt}{13mm}
\usepackage{setspace}
\sloppy
\setstretch{1.5}
\usepackage{indentfirst}
\parindent=1.25cm
\usepackage{caption}
\usepackage{subcaption}
\DeclareCaptionLabelSeparator{figSep}{.\quad}
\captionsetup[figure]{labelfont={normalfont}, name={Fig.}, labelsep=period}
\captionsetup[table]{labelfont={normalfont}, name={TABLE}, labelsep={newline}}
\title{The Employees Perspectives on the Adoption of DevOps}
\author{Ilona Dziurava, Egor Savchenko, Maksim Al Dandan} 
 
\affil{Innopolis University}

\begin{document}

\maketitle

\newpage

% USED: 7 8 9 10 11 12 13 14 15 17 18 20 21 int1 int2
% UNUSED: 16 19

\section{Introduction}

Development and Operations (DevOps) is a collection of methodologies and cultural principles that is used within the software development industry \cite{int1}. The stated benefits of DevOps includes increased corporate IT performance and productivity, lower software lifecycle costs, improved operational efficacy and efficiency, and higher-quality software products \cite{int1}. However, the implementation of DevOps remains a difficult issue \cite{int2}, \cite{8}, \cite{20}, \cite{14}. Moreover, limited understanding exists regarding practitioners' perspectives on effective DevOps adoption \cite{10}.

The objective of the research is to study the employees perspectives on the adoption of DevOps on a specific scenario.

\section{Literature Review}

% Definition
DevOps is an software quality-oriented approach aimed at reducing the gap between development and IT operations teams, even when they are spread across different locations \cite{17}. DevOps emphasizes building a collaborative culture and using automation to help team members interact effectively \cite{7}. The main aim is to boost the delivery of software changes by improving processes and encouraging continuous integration and delivery \cite{11}. Moreover, DevOps focuses on optimizing organizational structures and policies, responding to external pressures, refining release processes, meeting quality demands, and addressing socio-technical challenges \cite{7}.

% Benefits
DevOps integration is a crucial part of software development process, which helps to improve software security, sustainability, and performance. In \cite{7}, the authors found that incorporation of DevOps leads to security improvement, deployment predictability. Moreover, Azad et al. \cite{9} explored various success factors categorized within their study and concluded that the integration of DevOps methodologies notably enhances performance engineering, as well as automation in build and testing procedures. Additionally, the authors of \cite{12} and \cite{18} highlighted the beneficial effects of DevOps integration on improving software security and sustainability within companies. 

% Challenges
However, the adoption of DevOps is still a challenging task due to such factors as being in a transitional phase or demonstrating a cautious approach toward complete automation \cite{12}. Secondly, the complexity and range of skills required for successful DevOps implementation present significant obstacles. Thirdly, ineffective management of communication exacerbates coordination issues between development and operations teams \cite{7}. Security remains a prominent concern, as inadequate management may lead to significant data breaches and service disruptions in DevOps-based applications \cite{14}. Addressing these challenges requires innovative solutions. For example, the effectiveness of DevOps anomaly detection frameworks has been demonstrated in identifying and mitigating issues throughout the DevOps lifecycle \cite{13}.

\section{Research Design}

We have formulated our hypothesis based in the existing literature, leading to the selection of a deductive research design. Our investigation focuses on conducting a case study concerning the integration of DevOps within a particular company and collecting opinions of employees. To achieve this, the survey will be conducted both pre- and post-DevOps integration, necessitating a duration of approximately one year. This period will include the integration of DevOps practices along with a subsequent period of observation until distinguishable outcomes appear. Subsequently, a comprehensive analysis will arise, requiring an additional three-month period for research composition and documentation.

In order to gather data, a survey will be conducted to collect the viewpoints of employees. The survey will include all staff from the company. Utilization of a non-probability sampling method is used due to the targeted nature of the survey among employees within a specific company, thereby necessitating the application of purposive sampling. Company will be chosen among Innopolis Special Economic Zone. Subsequent to data collection, qualitative analysis techniques will be employed for thorough examination and interpretation.

\section{Anticipated Results}

The anticipated results of this study involve comparing two surveys to assess the impact of DevOps integration. Expected improvements include enhanced software development and deployment processes, improved communication and collaboration among teams, and an increase in the frequency of software releases. However, potential negative effects on job satisfaction are anticipated due to the challenges associated with employees adapting to DevOps practices.

\section{Discussion}

DevOps integration brings performance enhancement of the company but may be challenging for employees. Performance enhancement is expected to be similar to \cite{21}. Understanding employees' perspectives can provide insights into the challenges, benefits, and potential barriers to adopting DevOps practices. 

The results of the study may be used by companies, which are trying to integrate DevOps and do it more convinient for employees. Insights from employee perspectives can offer guidance on refining workflows, optimizing collaboration between teams, and identifying areas for improvement. This can lead to more efficient development processes and better integration of DevOps principles into the organization's culture. Without the insights from the research, organizations may allocate resources inefficiently in their efforts to implement DevOps practices. 

The study is constrained by its focus on a single company, which may limit the generalizability of our findings. Additionally, our evaluation lacks the use of metrics against previous studies, such as \cite{15}. Moreover, unlike \cite{16}, we have not developed a specific DevOps integration model based on our research findings. This limitation reduces the practical applicability of our results for guiding concrete implementation strategies in organizational contexts.

In conclusion, the research can inform industry professionals about real-world challenges related to DevOps adoption based on employees' perspective. 


\newpage
\printbibliography[heading=bibintoc,title={References}]

\end{document}
