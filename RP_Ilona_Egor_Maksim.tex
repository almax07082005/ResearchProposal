\documentclass[oneside,final,12pt,a4paper]{extreport}
\renewcommand{\thetable}{\Roman{table}}\usepackage{hyperref}
\usepackage{hyperref}

\hypersetup{
    colorlinks=true,
    linkcolor=blue,
    filecolor=magenta,      
    urlcolor=cyan,
    citecolor=cyan,
    pdftitle={Overleaf Example},
    pdfpagemode=FullScreen,
}
\usepackage[utf8]{inputenc}
\usepackage{graphicx}
\usepackage[affil-it]{authblk}
\usepackage[english]{babel}
\usepackage[backend=biber,style=ieee,autocite=inline]{biblatex}
\bibliography{ref.bib}
\DefineBibliographyStrings{english}{%
  bibliography = {References},}
\usepackage{blindtext}


\usepackage{amsmath,amsfonts}
\usepackage{url}
\usepackage{titlesec}
\usepackage{csquotes}
\usepackage{longtable}
\titleformat{\section}[hang]{\fontsize{20}{24}\selectfont\filcenter}{\Roman{section}}{1em}{}
\titleformat{\subsection}[hang]{\itshape}{\Alph{subsection}.}{1em}{}[]
\titleformat{\subsubsection}[runin]{\itshape}{\arabic{subsubsection})}{1em}{}[$:$]
\titlespacing{\subsubsection}{1em}{1em}{1em}
\titleformat{\paragraph}[runin]{\itshape}{\alph{paragraph})}{1em}{}[$:$\quad]
\titlespacing{\paragraph}{2em}{1em}{1em}
\usepackage{vmargin}
\setpapersize{A4}
\setmarginsrb{2.5cm}{2cm}{2cm}{2cm}{0pt}{10mm}{0pt}{13mm}
\usepackage{setspace}
\sloppy
\setstretch{1.5}
\usepackage{indentfirst}
\parindent=1.25cm
\usepackage{caption}
\usepackage{subcaption}
\DeclareCaptionLabelSeparator{figSep}{.\quad}
\captionsetup[figure]{labelfont={normalfont}, name={Fig.}, labelsep=period}
\captionsetup[table]{labelfont={normalfont}, name={TABLE}, labelsep={newline}}
\title{Research Proposal}
\author{Ilona, Egor, Maksim}
 
\affil{Innopolis University}

\begin{document}

\maketitle

\newpage

% USED: 7 8 9 11 12 13 14 18 20 21
% UNUSED: 10 15 16 17 19

\section{Introduction}

Development and Operations (DevOps) is a collection of methodologies and cultural principles that is used within the software development industry \cite{int1}. The stated benefits of DevOps includes increased corporate IT performance and productivity, lower software lifecycle costs, improved operational efficacy and efficiency, and higher-quality software products \cite{int1}. However, the implementation of DevOps remains a difficult issue \cite{int2}, \cite{8}, \cite{20}, \cite{14}. Moreover, limited understanding exists regarding practitioners' perspectives on effective DevOps adoption \cite{10}.

The objective of the research is to study the practical difficulties of the adoption of DevOps on a specific scenario.


\section{Literature Review}

% Definition
DevOps is an approach aimed at reducing the gap between development and IT operations teams, even when they are spread across different locations. DevOps emphasizes building a collaborative culture and using automation to help team members interact effectively \cite{7}. The main aim is to boost the delivery of software changes by improving processes and encouraging continuous integration and delivery \cite{11}. Moreover, DevOps focuses on optimizing organizational structures and policies, responding to external pressures, refining release processes, meeting quality demands, and addressing socio-technical challenges \cite{7}.

% Benefits
In \cite{7}, authors found out that integration of DevOps leads to security improvement, deployment predictability. Moreover, in \cite{9}, authors studied success factors by categories and found that DevOps integration improves perfomance engineering, build and test automation. In comparison with \cite{9}, authors of this article mentioned that DevOps integration helps companies to enhance software security and sustainability \cite{12}, \cite{18}. DevOps integration is a crucial part of software development process, which helps to improve software security, sustainability, and performance.

% Challenges
However, the adoption of DevOps is still a challenging task due to such factors as being in a transitional phase or demonstrating a cautious approach toward complete automation \cite{12}. Secondly, the complexity and range of skills required for successful DevOps implementation present significant obstacles. Thirdly, ineffective management of communication exacerbates coordination issues between development and operations teams \cite{7}. Security remains a prominent concern, as inadequate management may lead to significant data breaches and service disruptions in DevOps-based applications \cite{14}. Addressing these challenges requires innovative solutions. For example, the effectiveness of DevOps anomaly detection frameworks has been demonstrated in identifying and mitigating issues throughout the DevOps lifecycle \cite{13}.

\section{Research Design}

We are formulating the hypothesis based on the existing literature and then make an experiment, that is why we have decided to choose deductive research design.

We are going to produce case study on DevOps integration in a specific company. For producing such case study we need to collect data before DevOps integration and after. This will take about one year. After that, we will need to write up the research. This will take about 3 months.

% TODO: Specify metrics
% TODO: Specify a company
% TODO: Mention second survey

To collect data, we are going to conduct a survey to get opinions of employees. Moreover, we are planning to perform some metrics. We will use non-probability method because of conducting survey among employees of a specific company. Hence, we will use the purposive sampling. To analyze collected data we are going to use qualitative analysis.

\section{Anticipated Results}

% 1. Do you feel comfortable to work in this company?
% 2. Is it convinient to work with other departments of the company?
% 3. Is it convinient to work within your department?
% 4. Have productivity of the company got increased?

After conducting the second survey, we will anticipate comfortability get improved. Moreover, convinience of workers would get enhanced when working with other employees. Additionally, we will expect workers mention increasement of the company perfomance.

\section{Discussion}
% Research Gap
In \cite{11}, authors offers to do study in various areas of DevOps, as it is not thoroughly explored yet. In \cite{21}, authors not sure if DevOps actually was applied in selected cases.


\newpage
\printbibliography[heading=bibintoc,title={References}]

\end{document}
